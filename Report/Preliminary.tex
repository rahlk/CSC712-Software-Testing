\documentclass[12pt, journal, compsoc]{IEEEtran}
\usepackage{colortbl}
\usepackage{booktabs}
\usepackage{subcaption}
\usepackage{algorithm}
\usepackage{algorithmicx}
\usepackage{algpseudocode}
\usepackage{tabulary}
\usepackage{bigstrut}
\setlength\fboxsep{1pt}
\setlength\fboxrule{1pt}
\usepackage{multicol}
\bstctlcite{IEEEexample:BSTcontrol}
\usepackage[table]{xcolor}
\usepackage{picture}
\newcommand{\quart}[4]{\begin{picture}(100,3)
{\color{black}\put(#3,3){\circle*{4}}\put(#1,3){\line(1,0){#2}}}\end{picture}}
\usepackage{amsmath}
\usepackage{balance}
\usepackage{flushend}
\usepackage[english]{babel}
\usepackage{blindtext}
\usepackage{times}
\usepackage{cite}
\usepackage{hyperref}
\hypersetup{
  colorlinks = false,
  hidelinks = true
}
\setlength{\parindent}{0em}
\setlength{\parskip}{1em}


\begin{document}
  \markboth{CSC 712: Software Testing and Reliability. Fall, 
    2014}%
  {CSC 712 Software Testing and Reliability. Fall, 
    2014}
  
  \title{On Strategies for Improving Software Defect Prediction}
  \author{Rahul Krishna, 
    \IEEEauthorblockA{\normalsize {\textit{Dept. of Electrical and Computer 
          Engineering}\\
        North Carolina State University, Email: 
        \href{mailto:rkrish11@ncsu.edu}{{rkrish11@ncsu.edu}}}}}
  
  \IEEEcompsoctitleabstractindextext{%
    \begin{abstract}
      Programming inherently introduces defects into programs, as a result software systems can crash or fail to deliver an important functionality. It is very important to test a software throughly before it can be used. But an extensive testing can be prohibitively expensive or may take too much time to conduct This necessitates the use of automated software defect prediction tools. Although numerous machine learning algorithms are available to detect defects in software, but several factors undermine the accuracy of such algorithm. This paper uses Classification and Regression Trees (CART) and Random Forests to examines two approaches to counter the aforementioned problem. The first approach involves the use Synthetic Minority Oversampling Technique (also known as SMOTE). The second approach attempts to use a metaheursitic algorithm such as differential evolution to find the right set of parameters that can change the performance of the predictor. 
    \end{abstract}
    
    \begin{IEEEkeywords}
      Defect Prediction, Machine Learning, Differential Evolution, CART, Random Forest.
    \end{IEEEkeywords}}
\maketitle

\section{Introduction}

The rest of this paper is organized as follows--- Section \ref*{motivation} offers a small ilustration of the impact of SMOTE and tuning on the accuracy of the predictor. Section \ref*{back} highlights the underlying principles used in this paper. Section \ref{setup} presents the experimental setup followed by section \ref{expt} which presents the experimental results and discuss each one. Section \ref{concl} contains concluding remarks and finally section \ref{future} talks about the future work.
\section{Motivating Example}
\label{motivation}
\section{Background Notes}
\label{back}
\subsection{Defect Prediction}
\subsection{SMOTE}
\subsection{The Classifiers}
\subsection{Differential Evolution}
\begin{algorithm}[htbp!]
  
  \scriptsize
  \begin{algorithmic}[1]
    \Require $\mathit{np} = 10$, $f=0.75$, $cr=0.3$, $\mathit{life} = 5$, $\mathit{Goal} \in \{\mathit{pd},f,...\}$
    \Ensure $S_{best}$
    
    ~\\
    \Function{DE}{$\mathit{np}$, $f$, $cr$, $\mathit{life}$, $\mathit{Goal}$}
    \State $Population  \gets $ $InitializePopulation$($\mathit{np}$)   
    \State $S_{best} \gets $$GetBestSolution$($Population $)
    \While{$\mathit{life} > 0$}
    \State $NewGeneration \gets \emptyset$
    \For{$i=0 \to \mathit{np}-1$}
    \State $S_i \gets$ Extrapolate($Population [i], Population , cr, f$)
    \If {Score($S_i$)$\ge$Score($Population [i]$)}
    \State $NewGeneration$.$append$($S_i$)
    \Else
    \State $NewGeneration$.$append$($Population [i]$)
    \EndIf
    \EndFor
    \State $Population  \gets NewGeneration$
    \If{$\neg$ $Improve$($Population $)}
    \State $life -=1$
    \EndIf
    \State $S_{best} \gets$ $GetBestSolution$($Population $)
    \EndWhile
    \State \Return $S_{best}$
    \EndFunction
    \Function{Score}{$Candidate$}
    \State set tuned parameters according to $Candidate$
    \State $model \gets$$TrainLearner()$
    \State $result \gets$$TestLearner$($model$)   
    \State \Return$\mathit{Goal}(result)$  
    \EndFunction
    \Function{Extrapolate}{$old, pop, cr, f$}
    \State $a, b, c\gets threeOthers(pop,old)$  
    \State $newf \gets \emptyset$
    \For{$i=0 \to \mathit{np}-1$}
    \If{$cr < random()$}
    \State $newf$.$append$($old[i]$)
    \Else
    \If{typeof($old[i]$) == bool}
    \State $newf$.$append$(not $old[i]$)
    \Else
    \State $newf$.$append$(trim($i$,($a[i] + f * (b[i] - c[i]$)))) 
    \EndIf
    \EndIf
    \EndFor
    \State \Return $newf$
    \EndFunction
  \end{algorithmic} 
  \caption{Pesudocode for DE with Early Termination}
  \label{alg:DE}
\end{algorithm}
\section{Experimental Setup}
\subsection{Data Sets}
\begin{figure*}[htbp!]
  \renewcommand{\baselinestretch}{1}\begin{center}
    {\scriptsize
      \begin{tabular}{c|l|p{4.7in}}
        amc & average method complexity & e.g. number of JAVA byte codes\\\hline
        avg\_cc & average McCabe & average McCabe's cyclomatic complexity seen
        in class\\\hline
        ca & afferent couplings & how many other classes use the specific
        class. \\\hline
        cam & cohesion amongst classes & summation of number of different
        types of method parameters in every method divided by a multiplication
        of number of different method parameter types in whole class and
        number of methods. \\\hline
        cbm &coupling between methods &  total number of new/redefined methods
        to which all the inherited methods are coupled\\\hline
        cbo & coupling between objects & increased when the methods of one
        class access services of another.\\\hline
        ce & efferent couplings & how many other classes is used by the
        specific class. \\\hline
        dam & data access & ratio of the number of private (protected)
        attributes to the total number of attributes\\\hline
        dit & depth of inheritance tree &\\\hline
        ic & inheritance coupling &  number of parent classes to which a given
        class is coupled (includes counts of methods and variables inherited)
        \\\hline
        lcom & lack of cohesion in methods &number of pairs of methods that do
        not share a reference to an instance variable.\\\hline
        locm3 & another lack of cohesion measure & if $m,a$ are  the number of
        $methods,attributes$
        in a class number and $\mu(a)$  is the number of methods accessing an
        attribute, 
        then
        $lcom3=((\frac{1}{a} \sum_j^a \mu(a_j)) - m)/ (1-m)$.
        \\\hline
        loc & lines of code &\\\hline
        max\_cc & maximum McCabe & maximum McCabe's cyclomatic complexity seen
        in class\\\hline
        mfa & functional abstraction & number of methods inherited by a class
        plus number of methods accessible by member methods of the
        class\\\hline
        moa &  aggregation &  count of the number of data declarations (class
        fields) whose types are user defined classes\\\hline
        noc &  number of children &\\\hline
        npm & number of public methods & \\\hline
        rfc & response for a class &number of  methods invoked in response to
        a message to the object.\\\hline
        wmc & weighted methods per class &\\\hline
        \rowcolor{lightgray}
        defect & defect & Boolean: where defects found in post-release bug-tracking systems.
      \end{tabular}
    }
  \end{center}
  \caption{OO measures used in our defect data sets.  Last line is
    the dependent attribute (whether a defect is reported to  a
    post-release bug-tracking system).}\label{fig:ck}
\end{figure*}

\section{Experimental Results}
\begin{table*}[htbp!]
  \renewcommand{\baselinestretch}{1.25}
  \begin{subtable}{0.5\linewidth}
    
    {\tiny \begin{tabulary}{\linewidth}{|J|J|J|J|J|}
        \hline
        \textbf{Rank} & \textbf{Treatment} & \textbf{Med} & \textbf{IQR} & \\\hline
        1 &   RF &    41.0  &  3.0 & \quart{0}{7}{2}{-102} \\
        \hline  2 &   CART &    44.0  &  3.0 & \quart{10}{8}{10}{-102} \\
        \hline  3 & CART (SMOTE) &    70.0  &  2.0 & \quart{76}{5}{78}{-102} \\
        \hline  4 & RF (SMOTE) &    78.0  &  1.0 & \quart{97}{2}{99}{-102} \\[0.1cm]
        \hline \end{tabulary}} \caption{ant} \label{ant}
    
  \end{subtable}
  \begin{subtable}{0.5\linewidth}
    {\tiny \begin{tabulary}{\linewidth}{|J|J|J|J|J|}
        \hline
        \textbf{Rank} & \textbf{Treatment} & \textbf{Med} & \textbf{IQR} & \\\hline
        1 & RF &    39.0  &  1.0 & \quart{0}{4}{0}{-172} \\
        \hline  2 & CART &    43.0  &  2.0 & \quart{9}{9}{18}{-172} \\
        \hline  3 & CART (SMOTE) &    56.0  &  2.0 & \quart{72}{9}{77}{-172} \\
        \hline  4 & RF (SMOTE) &    60.0  &  2.0 & \quart{90}{9}{95}{-172} \\
        \hline \end{tabulary}}\caption{Camel} \label{Camel}
    
  \end{subtable}\\[0.2cm]
  
  \begin{subtable}{0.5\linewidth}
    {\tiny \begin{tabulary}{\linewidth}{|J|J|J|J|J|}
        \hline
        \textbf{Rank} & \textbf{Treatment} & \textbf{Med} & \textbf{IQR} & \\\hline
        1 & RF (SMOTE) &    0.0  &  0.0 & \quart{0}{0}{0}{1} \\
        1 & CART (SMOTE) &    15.0  &  15.0 & \quart{0}{26}{26}{1} \\
        \hline  2 &   RF &    50.0  &  1.0 & \quart{85}{2}{87}{1} \\
        \hline  3 &   CART &    56.0  &  1.0 & \quart{98}{1}{98}{1} \\
        \hline \end{tabulary}}\caption{Ivy} \label{Camel}
    
  \end{subtable}
  \begin{subtable}{0.5\linewidth}
    {\tiny \begin{tabulary}{\linewidth}{|J|J|J|J|J|}
        \hline
        \textbf{Rank} & \textbf{Treatment} & \textbf{Med} & \textbf{IQR} & \\\hline
        1 & RF &    0.0  &  0.0 & \quart{0}{0}{0}{1} \\
        1 & CART (SMOTE) &    84.0  &  1.0 & \quart{89}{1}{90}{1} \\
        1 & RF (SMOTE) &    88.0  &  1.0 & \quart{93}{1}{94}{1} \\
        1 & CART &    93.0  &  0.0 & \quart{99}{0}{99}{1} \\
        \hline \end{tabulary}}\caption{Jedit} \label{Camel}
    
  \end{subtable}\\[0.2cm]
  
  \begin{subtable}{0.5\linewidth}
    {\tiny \begin{tabulary}{\linewidth}{|J|J|J|J|J|}
        \hline
        \textbf{Rank} & \textbf{Treatment} & \textbf{Med} & \textbf{IQR} & \\\hline
        1 &   CART &    36.0  &  3.0 & \quart{0}{14}{0}{-166} \\
        1 &   RF &    40.0  &  4.0 & \quart{4}{19}{19}{-166} \\
        \hline  2 & RF (SMOTE) &    53.0  &  6.0 & \quart{71}{28}{80}{-166} \\
        2 & CART (SMOTE) &    54.0  &  4.0 & \quart{76}{19}{85}{-166} \\
        \hline \end{tabulary}}\caption{POI} \label{Camel}
    
  \end{subtable}
  \begin{subtable}{0.5\linewidth}
    {\tiny \begin{tabulary}{\linewidth}{|J|J|J|J|J|}
        \hline
        \textbf{Rank} & \textbf{Treatment} & \textbf{Med} & \textbf{IQR} & \\\hline
        1 & RF (SMOTE) &    2.0  &  2.0 & \quart{0}{4}{2}{0} \\
        \hline  2 & CART (SMOTE) &    14.0  &  5.0 & \quart{31}{12}{31}{0} \\
        \hline  3 & RF &    22.0  &  2.0 & \quart{48}{5}{51}{0} \\
        \hline  4 & CART &    41.0  &  2.0 & \quart{95}{4}{97}{0} \\
        \hline \end{tabulary}}\caption{Log4j} \label{Camel}
    
  \end{subtable}\\[0.2cm]
  
  \begin{subtable}{0.5\linewidth}
    {\tiny \begin{tabulary}{\linewidth}{|J|J|J|J|J|}
        \hline
        \textbf{Rank} & \textbf{Treatment} & \textbf{Med} & \textbf{IQR} & \\\hline
        1 & CART &    47.0  &  1.0 & \quart{0}{9}{0}{-418} \\
        \hline  2 & RF &    51.0  &  1.0 & \quart{36}{9}{36}{-418} \\
        2 & CART (SMOTE) &    50.0  &  4.0 & \quart{27}{36}{27}{-418} \\
        \hline  3 & RF (SMOTE) &    56.0  &  3.0 & \quart{72}{27}{81}{-418} \\
        \hline \end{tabulary}}\caption{Lucene} \label{Camel}
    
  \end{subtable}
  \begin{subtable}{0.5\linewidth}
    {\tiny \begin{tabulary}{\linewidth}{|J|J|J|J|J|}
        \hline
        \textbf{Rank} & \textbf{Treatment} & \textbf{Med} & \textbf{IQR} & \\\hline
        1 & RF &    51.0  &  0.0 & \quart{49}{0}{49}{-449} \\
        1 & CART &    53.0  &  0.0 & \quart{69}{0}{69}{-449} \\
        1 & CART (SMOTE) &    56.0  &  10.0 & \quart{0}{99}{99}{-449} \\
        1 & RF (SMOTE) &    56.0  &  1.0 & \quart{89}{10}{99}{-449} \\
        \hline \end{tabulary}}\caption{PBeans} \label{Camel}
    
  \end{subtable}\\[0.2cm]
  
  
  \begin{subtable}{0.5\linewidth}
    {\tiny \begin{tabulary}{\linewidth}{|J|J|J|J|J|}
        \hline
        \textbf{Rank} & \textbf{Treatment} & \textbf{Med} & \textbf{IQR} & \\\hline
        1 & CART (SMOTE) &    63.0  &  1.0 & \quart{0}{9}{9}{-609} \\
        \hline  2 & RF (SMOTE) &    68.0  &  2.0 & \quart{39}{20}{59}{-609} \\
        \hline  3 & CART &    70.0  &  2.0 & \quart{59}{20}{79}{-609} \\
        3 & RF &    70.0  &  2.0 & \quart{79}{20}{79}{-609} \\
        \hline \end{tabulary}}\caption{Velocity} \label{Camel}
    
  \end{subtable}
  \begin{subtable}{0.5\linewidth}
    {\tiny \begin{tabulary}{\linewidth}{|J|J|J|J|J|}
        \hline
        \textbf{Rank} & \textbf{Treatment} & \textbf{Med} & \textbf{IQR} & \\\hline
        1 & RF &    24.0  &  1.0 & \quart{0}{2}{0}{-58} \\
        \hline  2 & CART &    52.0  &  18.0 & \quart{53}{46}{71}{-58} \\
        2 & CART (SMOTE) &    59.0  &  2.0 & \quart{84}{5}{89}{-58} \\
        2 & RF (SMOTE) &    60.0  &  1.0 & \quart{92}{2}{92}{-58} \\
        \hline \end{tabulary}}\caption{Xalan} \label{Camel}
    
  \end{subtable}
  \caption{Performance scores (g values) for the data sets.}
\end{table*}
\subsection{SMOTE\textit{ing} improves Prediction Accuracy}
\subsection{Tuning Also improves Prediction Accuracy (?)}
\subsection{SMOTE\textit{ing}+Tuning improves Prediction Accuracy (?)}
\section{Conclusions}

\end{document}